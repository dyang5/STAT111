\documentclass[11pt]{article}
\usepackage{graphicx}
\usepackage{amsthm}
\usepackage{amsmath}
\usepackage{amssymb}
\usepackage[shortlabels]{enumitem}
\usepackage[margin=1in]{geometry}
\graphicspath{{img}}

\newenvironment{solution}
  {\renewcommand\qedsymbol{$\blacksquare$}\begin{proof}[Solution]}
  {\end{proof}}

\setlength\parindent{0pt}

\begin{document}

	\hrule
	\begin{center}
        \textbf{STAT111: Mathematical Statistics II}\hfill \textbf{Spring 2024}\newline

		{\Large Homework 14}

		David Yang
	\end{center}

\hrule

\vspace{1em}

\begin{enumerate}
    \item \textbf{The countries data set is from the CIA and reports Life Expectancy, Birth Rate, Real GDP per
    Capita and Education expenditure as a $\%$ of GDP. Find a model to predict life expectancy, or some
    transformation of life expectancy, from some or all of the other variables, or transformations of the
    other variables (interactions are possible, too). Show evidence that the Normal linear regression
    assumptions appear to be met. Make a $95\%$ prediction interval for the life expectancy in a country
    with a birth rate of 12 births per 1000 persons in a year, GDP per capita of $\$70,000$, and with $4\%$
    of GDP spent on education (close to the values for the USA).}
    
    \item \textbf{Suppose that the relation of family income to consumption is positive and roughly linear. Of those
    families in the 90th percentile of income, what proportion would you expect to be at or above the
    90th percentile of consumption: (a) exactly $50\%$, (b) less than $50\%$, of (c) more than $50\%$? Justify
    your answer.}
   
\end{enumerate}

\end{document}
