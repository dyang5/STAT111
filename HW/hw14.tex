\documentclass[11pt]{article}
\usepackage{graphicx}
\usepackage{amsthm}
\usepackage{amsmath}
\usepackage{amssymb}
\usepackage[shortlabels]{enumitem}
\usepackage[margin=1in]{geometry}
\usepackage{bm}

\graphicspath{{img}}

\newenvironment{solution}
  {\renewcommand\qedsymbol{$\blacksquare$}\begin{proof}[Solution]}
  {\end{proof}}

\setlength\parindent{0pt}

\begin{document}

	\hrule
	\begin{center}
        \textbf{STAT111: Mathematical Statistics II}\hfill \textbf{Spring 2024}\newline

		{\Large Homework 14}

		David Yang
	\end{center}

\hrule

\vspace{1em}

\begin{enumerate}
    \item \textbf{The countries data set is from the CIA and reports Life Expectancy, Birth Rate, Real GDP per
    Capita and Education expenditure as a $\%$ of GDP. Find a model to predict life expectancy, or some
    transformation of life expectancy, from some or all of the other variables, or transformations of the
    other variables (interactions are possible, too). Show evidence that the Normal linear regression
    assumptions appear to be met. Make a $95\%$ prediction interval for the life expectancy in a country
    with a birth rate of 12 births per 1000 persons in a year, GDP per capita of $\$70,000$, and with $4\%$
    of GDP spent on education (close to the values for the USA).}

    \begin{solution}
    See attached R Output for specific details. \\

    In my linear model, I use the birth rate and GDP variables to predict life expectancy (education was left out as it was not significant, and 
    none of the interaction variables with education were significant either). The assumptions for Normal linear regression were satisfied;
    there appears to be a relatively linear relationship between the variables, we see that the residuals are relatively normal, and 
    satisfy the homoscedasticity and independence conditions. \\

    The 95\% prediction interval I derived for the life expectancy under my model was approximately
    \[
      \boxed{[68.9, 82.6]}
    \]
    which seems on par with the data for countries such as the UK and the US.
    \end{solution}

    \item \textbf{Suppose that the relation of family income to consumption is positive and roughly linear. Of those
    families in the 90th percentile of income, what proportion would you expect to be at or above the
    90th percentile of consumption: (a) exactly $50\%$, (b) less than $50\%$, of (c) more than $50\%$? Justify
    your answer.}

    \begin{solution}
    For families in the 90th percentile of income, we expect the proportionto be at or above the
    90th percentile of consumption to be \boxed{\text{(c) more than $50\%$.}} 
    This follows from the fact that the relation of family income to consumption is positive and roughly linear, so
    we expect families with higher incomes to have higher levels of consumption.

    \end{solution}
   
	\newpage
	
    \item \textbf{The \textit{weighted} regression model assumes}
    \[
      \bm{Y} = \bm{X\beta} + \bm{\epsilon}, \, \, \, \bm{\epsilon} \sim N_n(\bm{0}, \sigma^{2}\bm{W}^{-1})
    \]
    \textbf{where $\bm{W}$ is a known, symmetric $n \times n$ positive definite matrix, and $\bm{X}$ is $n \times p$ and is also known.}

    \begin{enumerate}
      \item \textbf{Define $\hat{\bm{\beta}} = (\bm{X}^T \bm{WX})^{-1} \bm{X}^T \bm{WY}$. Show that}
      \[
        (\bm{Y} - \bm{X\beta})^T\bm{W}(\bm{Y} - \bm{X\beta}) = (\bm{Y} - \bm{X}\hat{\bm{\beta}})^T\bm{W}(\bm{Y} - \bm{X\hat{\bm{\beta}}}) + (\bm{\beta} - \hat{\bm{\beta}})\bm{X}^T\bm{WX}(\bm{\beta} - \hat{\bm{\beta}}).
      \]

      \begin{solution}
		We use our trick of adding and subtracting by the same constant in our left hand side expression, and then expanding. Note that
		\[
			(\bm{Y} - \bm{X\beta})^T\bm{W}(\bm{Y} - \bm{X\beta}) =  ((\bm{Y} - \bm{X}\hat{\bm{\beta}}) + (\bm{X}\hat{\bm{\beta}} - \bm{X\beta}))^T\bm{W}((\bm{Y} - \bm{X}\hat{\bm{\beta}}) + (\bm{X}\hat{\bm{\beta}} - \bm{X\beta}))
		\]        
		Expanding, we find that
		\begin{align*}
			(\bm{Y} - \bm{X\beta})^T\bm{W}(\bm{Y} - \bm{X\beta}) &=  ((\bm{Y} - \bm{X}\hat{\bm{\beta}}) + (\bm{X}\hat{\bm{\beta}} - \bm{X\beta}))^T\bm{W}((\bm{Y} - \bm{X}\hat{\bm{\beta}}) + (\bm{X}\hat{\bm{\beta}} - \bm{X\beta})) \\
			&= (\bm{Y} - \bm{X}\hat{\bm{\beta}})^T \bm{W} (\bm{Y} - \bm{X}\hat{\bm{\beta}}) + (\bm{X}\hat{\bm{\beta}} - \bm{X}\bm{\beta})^T \bm{W} (\bm{X}\hat{\bm{\beta}} - \bm{X}\bm{\beta}) \\
			&+ (\bm{Y} - \bm{X}\hat{\bm{\beta}})^T \bm{W} (\bm{X}\hat{\bm{\beta}} - \bm{X}\bm{\beta}) + (\bm{X}\hat{\bm{\beta}} - \bm{X}\bm{\beta})^T \bm{W} (\bm{Y} - \bm{X}\hat{\bm{\beta}})
		\end{align*}
		Note that the latter two terms are both $1 \times 1$, so they are equal.
		To simplify the above expression, we work with the term
		\begin{align*}
			(\bm{X}\hat{\bm{\beta}} - \bm{X}\bm{\beta})^T \bm{W} (\bm{Y} - \bm{X}\hat{\bm{\beta}}) &= (\hat{\bm{\beta}} - \bm{\beta})^T\bm{X}^T \bm{W} (\bm{Y} - \bm{X}\hat{\bm{\beta}}) \\
			&= (\hat{\bm{\beta}} - \bm{\beta})^T (\bm{X}^T\bm{WY} - \bm{X}^T\bm{WX}\hat{\bm{\beta}}) \\
			&=  (\hat{\bm{\beta}} - \bm{\beta})^T (\bm{X}^T\bm{WY} - \bm{X}^T\bm{WX}((\bm{X}^T\bm{WX})^{-1}\bm{X}^T \bm{WY})) \\
			&=  (\hat{\bm{\beta}} - \bm{\beta})^T (\bm{X}^T\bm{WY} - \bm{X}^T \bm{WY}) \\
			&= 0. 
		\end{align*}
		Thus, we simplify to get that 
		\[
			(\bm{Y} - \bm{X\beta})^T\bm{W}(\bm{Y} - \bm{X\beta}) = (\bm{Y} - \bm{X}\hat{\bm{\beta}})^T\bm{W}(\bm{Y} - \bm{X\hat{\bm{\beta}}}) + (\bm{\beta} - \hat{\bm{\beta}})\bm{X}^T\bm{WX}(\bm{\beta} - \hat{\bm{\beta}})
		\]
		as desired.
      \end{solution}
      \item \textbf{Explain why (a) implies that $\hat{\bm{\beta}}$ is the MLE for $\bm{\beta}$ but not the least squares estimate.}
	  \begin{solution}
		Note that the above expression is minimized for $\bm{\beta}$ when $\beta = \hat{\bm{\beta}}$, in which case the second term equals zero (if $\beta \neq \hat{\bm{\beta}}$ , then the second term is positive).
		Thus, $\hat{\bm{\beta}}$ is the MLE for $\bm{\beta}$. However, it is not the least squares estimate.
	  \end{solution}

	  \item \textbf{Define $\bm{Y}^* = (\bm{W}^{\frac{1}{2}})\bm{Y}$ and $\bm{X}^* = (\bm{W}^{\frac{1}{2}})\bm{X}$, where $\bm{W}^{\frac{1}{2}}$ is a symmetric square root matrix.
      Show this transforms the problem to the standard, unweighted regression model, and the usual Least Squares formula with $\bm{Y}^*$ and $\bm{X}^*$ implies the same value $\hat{\bm{\beta}}$.}

	  \begin{solution}
	Note that with  $\bm{Y}^* = (\bm{W}^{\frac{1}{2}})\bm{Y}$ and $\bm{X}^* = (\bm{W}^{\frac{1}{2}})\bm{X}$, we get that
	  \[
		\hat{\bm{\beta}} = (\bm{X}^T \bm{WX})^{-1} \bm{X}^T \bm{WY} = ((\bm{X}^*)^T\bm{X}^*)^{-1} (\bm{X}^*)^T \bm{Y}^*.
	  \]
	  This directly matches the form of our standard unweighted regression model, and we get the formula
	  \[
		(\bm{Y}^* - \bm{X^*\beta})^T(\bm{Y}^* - \bm{X^*\beta}) = (\bm{Y}^* - \bm{X}^*\hat{\bm{\beta}})^T(\bm{Y}^* - \bm{X^*\hat{\bm{\beta}}}) + (\hat{\bm{\beta}} - \bm{\beta})(\bm{X}^*)^T\bm{X}^*(\hat{\bm{\beta}} - \bm{\beta}).
	  \]
	  which is also our typical least squares formula. This implies the same value for $\hat{\bm{\beta}}$. 
	  \end{solution}
	  
      \item \textbf{Find the sampling distribution of $\hat{\bm{\beta}}$ (given $\bm{\beta}$ and $\sigma^2$).}
	  \begin{solution}
		Note that $\hat{\bm{\beta}} = (\bm{X}^T \bm{WX})^{-1} \bm{X}^T \bm{WY}$ and  $\bm{Y} \sim N_n(\bm{X\beta}, \sigma^2\bm{W}^{-1})$. It follows that
		\begin{align*}
			\mathbb{E}\left[\hat{\bm{\beta}} \right] &= (\bm{X}^T \bm{WX})^{-1} \bm{X}^T \bm{W}  \mathbb{E}\left[\bm{Y} \right] \\
			&= (\bm{X}^T \bm{WX})^{-1} \bm{X}^T \bm{W} \bm{X\beta} \\
			&= \bm{\beta}.
		\end{align*}

		Similarly,
		\begin{align*}
			\mathrm{Var} \left[\hat{\bm{\beta}} \right] &= \mathrm{Var} \left[ (\bm{X}^T \bm{WX})^{-1} \bm{X}^T \bm{W} \bm{Y} \right] \\
			&= (\bm{X}^T \bm{WX})^{-1} \bm{X}^T \bm{W} \,\mathrm{Var} \left[Y \right] ((\bm{X}^T \bm{WX})^{-1} \bm{X}^T \bm{W})^T \\
			&= (\bm{X}^T \bm{WX})^{-1} \bm{X}^T \bm{W} \sigma^2 \bm{W}^{-1} ((\bm{X}^T \bm{WX})^{-1} \bm{X}^T \bm{W})^T \\
			&= \sigma^2 (\bm{X}^T \bm{WX})^{-1} \bm{X}^T \bm{W} \bm{W}^{-1}\bm{W}^T\bm{X} ((\bm{X}^T \bm{WX})^{-1})^T \\\
			&= \sigma^2 (\bm{X}^T \bm{WX})^{-1} \bm{X}^T \bm{W}\bm{X} ((\bm{X}^T \bm{WX})^{-1})^T \\
			&= \sigma^2 (\bm{X}^T \bm{WX})^{-1}.
		\end{align*}
		Thus, we find that the sampling distribution for $\bm{\hat{\beta}}$ is
		\[
			\boxed{\bm{\hat{\beta}} \mid \bm{\beta}, \sigma^2 \sim N_p(\bm{\beta}, \sigma^2(\bm{X}^T\bm{WX})^{-1})}. \, \qedhere
		\]
	  \end{solution}
		
      \item \textbf{As an example where this would be useful, imagine a situation where the Normal linear model is appropriate, 
      but where each independent response value $Y_i$ is the average of $m_i$ independent measurements. For example, the logs of the median
      selling prices in the home sales data are roughly the averages of individual log-selling prices for each municipality, with different numbers
      of houses sold (e.g. $m_i = 1767$ for Upper Darby and $m_i = 17$ for Swarthmore). Describe the appropriate $\bm{W}$ matrix.}
    \end{enumerate}
\end{enumerate}

\end{document}
