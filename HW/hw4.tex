\documentclass[11pt]{article}
\usepackage{graphicx}
\usepackage{amsthm}
\usepackage{amsmath}
\usepackage{amssymb}
\usepackage[shortlabels]{enumitem}
\usepackage[margin=1in]{geometry}

\newenvironment{solution}
  {\renewcommand\qedsymbol{$\blacksquare$}\begin{proof}[Solution]}
  {\end{proof}}

\setlength\parindent{0pt}

\begin{document}

	\hrule
	\begin{center}
        \textbf{STAT111: Mathematical Statistics II}\hfill \textbf{Spring 2024}\newline

		{\Large Homework 3}

		David Yang
	\end{center}

\hrule

\vspace{1em}

\begin{enumerate}
    \item \textbf{Suppose $V_1$ and $V_2$ are independent $\mathrm{Gamma}(1, \lambda)$ random variables that represent waiting
    times in a Poisson process with rate $\lambda$ events per unit time. Let $X = V_1$ be the time of the
    first event and let $Y = V_1 + V_2$ be the time of the second event.}
  

    \item \textbf{Suppose $X$ and $Y$ have joint pdf $f_{xy}(x, y) = I(0 < x < 1, -x < y < x)$.}
    \begin{enumerate}[a)]
      \item \textbf{Explain how you can tell, without finding the marginal densities, that the conditional densities are Uniform.
      Write out the conditional densities $f_{x \mid y}(x \mid y)$ and $f_{y \mid x}(y \mid x)$.}
      \item \textbf{Explain how you can tell, without finding the marginal densities, that $X$ and $Y$ are not independent. Find the marginal pdf's $f_x(x)$ and $f_y(y)$ 
      and verify that $f_{xy}(x, y) \neq f_x(x)f_y(y)$.}
      \item \textbf{Show that $X$ and $Y$ are uncorrelated.}
    \end{enumerate}

    \begin{enumerate}[a)]
      \item \textbf{Suppose $X_1$ and $X_2$ are Bernoulli random variables with expectations $p_1$ and $p_2$. Show that $X_1$ and $X_2$ are independent if and only if they are uncorrelated.
      This shows the Bernoulli distribution is special like the multivariate Normal distribution in that uncorrelated implies independence.}
      \item \textbf{Suppose $Y = X_1 + X_2$ with $X_1$ and $X_2$ independent. If you learn that $Y$ and $X_1$ are
      Normal variables, prove that $X_2$ is also a Normal random variable.}
    \end{enumerate}
    
\end{enumerate}

\end{document}
