\documentclass[11pt]{article}
\usepackage{graphicx}
\usepackage{amsthm}
\usepackage{amsmath}
\usepackage{amssymb}
\usepackage[shortlabels]{enumitem}
\usepackage[margin=1in]{geometry}

\newenvironment{solution}
  {\renewcommand\qedsymbol{$\blacksquare$}\begin{proof}[Solution]}
  {\end{proof}}

\setlength\parindent{0pt}

\begin{document}

	\hrule
	\begin{center}
        \textbf{STAT111: Mathematical Statistics II}\hfill \textbf{Spring 2024}\newline

		{\Large Homework 1}

		David Yang
	\end{center}

\hrule

\vspace{1em}

\begin{enumerate}
    \item \textbf{Evaluate the integral}
    \[
        \int_0^{\infty} x^5 e^{-2x} \, dx
    \] 
    \textbf{using integration by recognition. That is, recognize this function as proportional to a standard pdf 
    and identify the constant multiplier needed to make the integral equal $1$. Then take the reciprocal of that constant.}

    \begin{solution}
    The integrand is the kernel of a $\mathrm{Gamma}(6, 2)$ random variable. \\

    Using this fact, we know that the PDF integrates to $1$, i.e.
    \[
        \int_0^{\infty} \frac{2^6}{\Gamma(6)}x^5 e^{-2x} \, dx = 1.
    \] 
    
    Solving for the integral we want to evaluate, we find that
    \[
        \int_0^{\infty} x^5 e^{-2x} \, dx = \frac{\Gamma(6)}{2^6} = \frac{(6-1)!}{64} = \boxed{\frac{15}{8}}. \qedhere
    \] 
    \end{solution}

\newpage

    \item \textbf{Suppose that $X \sim \mathrm{Gamma}\left(\alpha, \frac{\alpha}{\mu}\right)$ is parameterized so that the mean is $\mu$.}
    
    \begin{enumerate}[a)]
        \item \textbf{Identify the mode of the pdf for $X$ as a function of $\alpha$ and $\mu$. That is, for what value of $x$ is $f_x(x)$ (or $\ln f_x(x)$) maximized?} 
    
        \begin{solution}
        $X \sim \mathrm{Gamma}\left(\alpha, \frac{\alpha}{\mu}\right)$ has pdf 
        \[
            f_X(x) = \frac{\left( \frac{\alpha}{\mu} \right)^{\alpha} }{\Gamma(\alpha)} x^{\alpha - 1} e^{-\left( \frac{\alpha}{\mu} \right)x}.
        \]

        $f_X(x)$ is maximized when $\ln(f_X(x))$ is maximized. For convenience, let $\ell(x)$ denote $\ln(f_X(x))$. Note that 
        \[
            \ell(x) = \ln(f_X(x)) = \ln \left( \frac{\left( \frac{\alpha}{\mu} \right)^{\alpha} }{\Gamma(\alpha)} \right) + (\alpha - 1) \ln(x) - \left( \frac{\alpha}{\mu} \right)x. 
        \]
        We take the derivative of $\ell(x)$ and set it to $0$ to solve for the maximum:
        \[
            \ell^{\prime}(x) = \frac{\alpha - 1}{x} - \frac{\alpha}{\mu}.
        \]

        Note that $\ell^{\prime}(x) = 0$ when $\frac{\alpha - 1}{x} - \frac{\alpha}{\mu}$. Solving for $x$, we find that $x = \frac{\mu(\alpha - 1)}{\alpha}$. \\

        However, note that this is the mode only when $\alpha > 1$ (for which $x > 0$); when $\alpha \leq 1$, the mode occurs at $x=0$. Thus, the mode of the pdf for $X$ as a function of $\alpha$ and $\mu$ is

            \[
                \boxed{
                    x = \begin{cases}
                        0 &\text{ if } \alpha \leq 1  ;\\
                        \frac{\mu(\alpha - 1)}{\alpha}  &\text{ if } \alpha > 1
                    \end{cases}
                } \qedhere
            \]
        \end{solution}

        \item \textbf{Let $Y = \frac{1}{X}$, so that $Y$ follows a reciprocal-Gamma$\left( \alpha, \frac{\alpha}{\mu} \right)$ distribution. Find the pdf for $Y$,
        and identify its mode as a function of $\alpha$ and $u$.}

        \begin{solution}
        Let $Y = \frac{1}{X}$. By the change of variables formula for pdfs, we know that
        \[
            f_Y(y) = f_X(g^{-1}(y)) \left| \frac{d}{dy} \left( g^{-1}(y) \right) \right|
        \]
        where $g^{-1}(y) = \frac{1}{y}$ and $\frac{d}{dy} \left( g^{-1}(y) \right) = \frac{d}{dy} \left( \frac{1}{y} \right) = -\frac{1}{y^2}$. \\
        
        Plugging these back into our above formula, we have that
        \begin{align*}
            f_Y(y) &= f_X(g^{-1}(y)) \left| \frac{d}{dy} \left( g^{-1}(y) \right) \right| \\
            &= \left[ \frac{\left( \frac{\alpha}{\mu} \right)^{\alpha} }{\Gamma(\alpha)} \left( \frac{1}{y} \right)^{\alpha - 1} e^{-\left( \frac{\alpha}{\mu} \right)\left(\frac{1}{y} \right) } \right] \left( \frac{1}{y^2} \right) \\
            &= \frac{\left( \frac{\alpha}{\mu} \right)^{\alpha} }{\Gamma(\alpha)} y^{-\alpha - 1} e^{- \frac{\alpha}{\mu y} }. 
        \end{align*}

        Thus, the pdf for $Y$ is 
        \[
            \boxed{f_Y(y) = \frac{\left( \frac{\alpha}{\mu} \right)^{\alpha} }{\Gamma(\alpha)} y^{-\alpha - 1} e^{- \frac{\alpha}{\mu y} }}.
        \]

        To identify the mode of $y$, we can once again work to maximize the log of the pdf for $Y$, denoted $\ell(y)$ for convenience. Note that
        \[
            \ell(y) = \ln \left( \frac{\left( \frac{\alpha}{\mu} \right)^{\alpha} }{\Gamma(\alpha)} \right) + (-\alpha - 1) \ln(y) - \frac{\alpha}{\mu y}.
        \]

        Taking the derivative of $\ell(y)$, we find
        \[
            \ell^{\prime}(y) = \frac{-\alpha - 1}{y} + \frac{\alpha}{\mu y^2}.
        \]

        The mode occurs when $\ell^{\prime}(y) = 0$, or when $\frac{\alpha + 1}{y} + \frac{\alpha}{\mu y^2}$. Solving for $y$, we find that the mode is at
        \[
            \boxed{y = \frac{\alpha}{\mu(\alpha + 1)}}
        \]
        for all $\alpha > 0$.
        \end{solution}
    \end{enumerate}

    \item \textbf{Let $F(x) = \frac{x}{x+2} I_{(x > 0)}$.}
    
    \begin{enumerate}[a)]
        \item \textbf{Show that $F_x(x)$ is a CDF and find the corresponding pdf.}
        
        \begin{solution}
        First, note that $F_x(x)$ is nondecreasing as 
        \[
            F^{\prime}_x(x) = \frac{2}{(x+2)^2} > 0
        \] 
        for any $x$. \\

        Furthermore,
        \[
            \lim_{x \to -\infty} F(x) = 0
        \]
        as $F(x) = 0$ for any $x \leq 0$. \\
        
        Finally, note that  
        \[
            \lim_{x \to \infty} F(x) = \lim_{x \to \infty} \frac{x}{x+2} = 1.
        \]

        Thus, $F_x(x)$ satisfies all the properties of a CDF. The corresponding pdf for $X$ is simply $F^{\prime}_x(x)$, or
        \[
            \boxed{f(x) = \frac{2}{(x+2)^2} I_{(x > 0)}}. \qedhere
        \]
        \end{solution}

        \item \textbf{Identify these as the CDF and pdf for an $F^*$ random variable (give the parameter values $a$, $b$, and $c$).}
        
        \begin{solution}
        Recall that if $X \sim F^*(a, b, c)$, then 
        \[
            f_X(x) \propto \frac{x^{a - 1}}{(c+x)^{a+b}} I(x > 0).
        \]
        Comparing the derived PDF $f_x = \frac{2}{(x+2)^2} I_{(x > 0)}$ to an $F^*$ distribution kernel, we find that $a = 1$, $b = 1$, and $c = 2$, so
        \[
            \boxed{X \sim F^*(1, 1, 2)}. \qedhere
        \]
        \end{solution}
        
        \item \textbf{For a random variable $X$ that follows this $F^*$ distribution, represent $X$ in terms of two independent Gamma random variables and a positive constant $c$.
        Use this representation to identify the distributions of $Y = \frac{1}{X}$ and of $R = \frac{X}{2+X}$.}

        \begin{solution}
        Recall that $F^*(a, b, c) = c \frac{V_1}{V_2}$ where $V_1 \sim \mathrm{Gamma}(a, 1)$ and $V_2 \sim \mathrm{Gamma}(b, 1)$. \\
        
        It follows that since $X \sim F^*(1, 1, 2)$, 
        \[
            \boxed{X = 2 \frac{V_1}{V_2}}
        \] where $V_1 \sim \mathrm{Gamma(1, 1)}$ and $V_2 \sim \mathrm{Gamma(1, 1)}$ are independent.

        Furthermore, note that for $Y = \frac{1}{X} = \frac{1}{2} \frac{V_2}{V_1}$, we have that
        \[
            \boxed{Y \sim F^*\left( 1, 1, \frac{1}{2} \right)} 
        \]
        and for $R = \frac{X}{2+X}$,
        \[
            \boxed{R \sim \mathrm{Beta}(1, 1)}
        \]
        by properties of the $F^*$ distribution. 
        \end{solution}

    \end{enumerate}

    \item \textbf{Suppose $X \mid \theta \sim \mathrm{Poisson}(\theta)$, with $\theta \sim \mathrm{Gamma}(\alpha, \lambda)$. Find the marginal pmf for $X$ by integrating $\theta$ out of the joint pmf/pdf.
    Show that this is a Negative Binomial distribution that represents the count of successes at the time 
    of our $\alpha$th failure (if $\alpha$ happens to be an integer) and identify the success probability.}


\end{enumerate}

\end{document}
