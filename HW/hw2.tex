\documentclass[11pt]{article}
\usepackage{graphicx}
\usepackage{amsthm}
\usepackage{amsmath}
\usepackage{amssymb}
\usepackage[shortlabels]{enumitem}
\usepackage[margin=1in]{geometry}

\newenvironment{solution}
  {\renewcommand\qedsymbol{$\blacksquare$}\begin{proof}[Solution]}
  {\end{proof}}

\setlength\parindent{0pt}

\begin{document}

	\hrule
	\begin{center}
        \textbf{STAT111: Mathematical Statistics II}\hfill \textbf{Spring 2024}\newline

		{\Large Homework 1}

		David Yang
	\end{center}

\hrule

\vspace{1em}

\begin{enumerate}
    \item \textbf{Evaluate the integral}
    \[
        \int_0^{\infty} x^5 e^{-2x} \, dx
    \] 
    \textbf{using integration by recognition. That is, recognize this function as proportional to a standard pdf 
    and identify the constant multiplier needed to make the integral equal $1$. Then take the reciprocal of that constant.}

    \begin{solution}
    The integrand is the kernel of a $\mathrm{Gamma}(6, 2)$ random variable. \\

    Using this fact, we know that the PDF integrates to $1$, i.e.
    \[
        \int_0^{\infty} \frac{2^6}{\Gamma(6)}x^5 e^{-2x} \, dx = 1.
    \] 
    
    Solving for the integral we want to evaluate, we find that
    \[
        \int_0^{\infty} x^5 e^{-2x} \, dx = \frac{\Gamma(6)}{2^6} = \frac{(6-1)!}{64} = \boxed{\frac{15}{8}}. \qedhere
    \] 
    \end{solution}

\newpage

    \item \textbf{Suppose that $X \sim \mathrm{Gamma}\left(\alpha, \frac{\alpha}{u}\right)$ is parameterized so that the mean is $\mu$.}
    
    \begin{enumerate}[a)]
        \item \textbf{Identify the mode of the pdf for $X$ as a function of $\alpha$ and $u$. That is, for what value of $x$ is $f_x(x)$ (or $\ln f_x(x)$) maximized?} 
    
        \begin{solution}
        $X \sim \mathrm{Gamma}\left(\alpha, \frac{\alpha}{u}\right)$ has pdf 
        \[
            f_X(x) = \frac{\left( \frac{\alpha}{u} \right)^{\alpha} }{\Gamma(\alpha)} x^{\alpha - 1} e^{-\left( \frac{\alpha}{u} \right)x}.
        \]

        $f_X(x)$ is maximized when $\ln(f_X(x))$ is maximized. For convenience, let $\ell(x)$ denote $\ln(f_X(x))$. Note that 
        \[
            \ell(x) = \ln(f_X(x)) = \ln \left( \frac{\left( \frac{\alpha}{\mu} \right)^{\alpha} }{\Gamma(\alpha)} \right) + (\alpha - 1) \ln(x) - \left( \frac{\alpha}{\mu} \right)x. 
        \]
        We take the derivative of $\ell(x)$ and set it to $0$ to solve for the maximum:
        \[
            \ell^{\prime}(x) = \frac{\alpha - 1}{x} - \frac{\alpha}{\mu}.
        \]

        Note that $\ell^{\prime}(x) = 0$ when $\frac{\alpha - 1}{x} - \frac{\alpha}{\mu}$. Solving for $x$, we find that $x = \frac{\mu(\alpha - 1)}{\alpha}$. 
        \end{solution}


        \item \textbf{Let $Y = \frac{1}{X}$, so that $Y$ follows a reciprocal-Gamma$\left( \alpha, \frac{\alpha}{u} \right)$ distribution. Find the pdf for $Y$,
        and identify its mode as a function of $\alpha$ and $u$.}
    \end{enumerate}

    \item \textbf{Let $F(x) = \frac{x}{x+2} I_{(x > 0)}$.}
    
    \begin{enumerate}[a)]
        \item \textbf{Show that $F_x(x)$ is a CDF and find the corresponding pdf.}
        \item \textbf{Identify these as the CDF and pdf for an $F^*$ random variable (give the parameter values $a$, $b$, and $c$).}
        \item \textbf{For a random variable $X$ that follows this $F^*$ distribution, represent $X$ in terms of two independent Gamma random variables and a positive constant $c$.
        Use this representation to identify the distributions of $Y = \frac{1}{X}$ and of $R = \frac{X}{2+X}$.}
    \end{enumerate}

    \item \textbf{Suppose $X \mid \theta \sim \mathrm{Poisson}(\theta)$, with $\theta \sim \mathrm{Gamma}(\alpha, \lambda)$. Find the marginal pmf for $X$ by integrating $\theta$ out of the joint pmf/pdf.
    Show that this is a Negative Binomial distribution that represents the count of successes at the time 
    of our $\alpha$th failure (if $\alpha$ happens to be an integer) and identify the success probability.}


\end{enumerate}

\end{document}
