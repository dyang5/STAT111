\documentclass[11pt]{article}
\usepackage{graphicx}
\usepackage{amsthm}
\usepackage{amsmath}
\usepackage{amssymb}
\usepackage[shortlabels]{enumitem}
\usepackage[margin=1in]{geometry}

\newenvironment{solution}
  {\renewcommand\qedsymbol{$\blacksquare$}\begin{proof}[Solution]}
  {\end{proof}}

\setlength\parindent{0pt}

\begin{document}

	\hrule
	\begin{center}
        \textbf{STAT111: Mathematical Statistics II}\hfill \textbf{Spring 2024}\newline

		{\Large Homework 3}

		David Yang
	\end{center}

\hrule

\vspace{1em}

\begin{enumerate}
    \item \textbf{Suppose a planet has $m$ days in a year, and life forms have equal probability of being hatched on any of these days. For a random group of $n$
    lifeworms, find the expected proportion of the $m$ possible hatch days that are represented. Find the value when $m = 365$ and $n = 365$ (hint: use indicator variables; the answer is close to $1 - e^{-1}$.)}

    \item \textbf{Suppose $X$ has pmf $P(X = k) = \frac{c}{(1+|k|)^2}$ for $k = 0, \pm 1, \pm 2, \dots$. The constant
    $c = (2\psi^{\prime}(1) - 1)^{-1}$, where $\psi^{\prime}(\alpha) = \frac{d^2}{d\alpha^2} \log \Gamma(\alpha)$ is the trigamma function.
    Explain why $E(X)$ is not $0$, despite the symmetry of this pmf.}
    
    \item \textbf{Suppose $X \sim \mathrm{Gamma}(\alpha, \lambda)$, with pdf $f_x(x) = \frac{\lambda^\alpha}{\Gamma(\alpha)}x^{\alpha - 1}e^{-\lambda x}I(x > 0)$ for $\alpha > 0$ and $\lambda > 0$.}
    
    \begin{enumerate}[a)]
      \item \textbf{Find an expression for $E(X^k)$, for $k = 1, 2, \dots$, using integration by recognition.}
      \item \textbf{$Y = \frac{1}{X}$ follows a reciprocal Gamma distribution. Find $E(Y)$, $E(Y^2)$, and $\mathrm{Var} \left[Y \right]$, first using integration by recognition with the pdf found in HW2, and again
      using LOTUS with the pdf for $X$. Be sure to say if there are conditions when these are not defined.}
    \end{enumerate}

    \item \textbf{Suppose $V \sim \mathrm{Gamma}(b, 1)$ and $X \mid V \sim \mathrm{Gamma}(a, V)$.}
    
    \begin{enumerate}[a)]
      \item \textbf{Show that $X \sim F^*(a, b, 1)$.}
      \item \textbf{Use the laws of total expectation and variance to find the mean and variance of the $F^*(a, b, 1)$ distribution. Be sure to say if there are conditions when these are not defined.}
    \end{enumerate}
\end{enumerate}

\end{document}
