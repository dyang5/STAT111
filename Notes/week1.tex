\section{Discrete Probability Distributions}

\setcounter{subsection}{1}
\subsection{Indicator Variables}

\begin{enumerate}
    \item \textbf{TBD}

        
    \item \textbf{In the caps example, let $X$ represent the number of graduates who retrieve their own cap. Explain why the distribution of $X$ is approximately
    Poisson$(1)$ when $n$ is large (see 1(c)). Compare the exact probabilities of $X = 0$ and $X=1$ to the corresponding Poisson probabilities when $n=5$.}

    Note that the probability that a given graduate receives their own cap (assuming nothing about all other graduate cap arrangements) is $p = \frac{1}{n}$. Though we are essentially sampling without replacement when throwing all the graduate caps in the air and retrieving them at random, we know from problem 1(b) that the distribution $X$ for the number of graduates who receive their own cap converges to $\mathrm{Binom}(n, p)$ when $n$ is large. \\

    Furthermore, note that $\lambda = np = n\frac{1}{n} = 1$ is fixed. Consequently, the limit of $P(X = x)$ as $n$ is large, by problem 1(c), is simply $\mathrm{Poisson}(\lambda) = \mathrm{Poisson}(1)$. We conclude that the distribution of $X$ is approximately $\mathrm{Poisson}(1)$ when $n$ is large. \\

    Using the approximation $X \sim \mathrm{Poisson}(1)$ for large $n$, we can approximate the probability of a derangement
    \[ 
        f_x(x=0, \lambda = 1) = \frac{1^0 e^{-1}}{0!} = \frac{1}{e}
    \]
    and the probability that exactly one graduate gets their own cap:
    \[ 
        f_x(x=1, \lambda = 1) = \frac{1^1 e^{-1}}{1!} = \frac{1}{e}.
    \]

    We can also compare these approximations with the exact probabilities of $X=0$ and $X=1$ for $n=5$. Note that the probability of a derangement for $n=5$ can be calculated using the formula from 2(b):
    \begin{align*}
        P(X = 0)&= 1 - \left[ 5 \cdot \frac{1}{5} - \binom{5}{2}\frac{1}{5} \cdot \frac{1}{4} + 
        \binom{5}{3}\frac{1}{5} \cdot \frac{1}{4} \cdot \frac{1}{3} - \binom{5}{4}\frac{1}{5} \cdot \frac{1}{4} \cdot \frac{1}{3} \cdot \frac{1}{2}
        + \binom{5}{5}\frac{1}{5} \cdot \frac{1}{4} \cdot \frac{1}{3} \cdot \frac{1}{2} \cdot \frac{1}{1}\right] \\
        &= \frac{11}{30} \approx 0.3\overline{6}
    \end{align*}

    Note that the probability that $X=1$ can be calculated by ``picking'' the graduate to get their own cap (of which there are $5$ possibilites),
    multiplying this by the probability that the chosen graduate gets their own hat ($\frac{1}{5}$), and then multiplying this by the probability of a derangement with $n=4$ (each of the
    remaining four graduates does not get their own cap):
    \begin{align*}
        P(X = 1) &= 5 \cdot \frac{1}{5} \cdot P(\text{derangement for } n = 4 \text{ students}) \\
        &= 1 - \left[ 4 \cdot \frac{1}{4} - \binom{4}{2}\frac{1}{4} \cdot \frac{1}{3} + 
        \binom{4}{3}\frac{1}{4} \cdot \frac{1}{3} \cdot \frac{1}{2} - \binom{4}{4}\frac{1}{4} \cdot \frac{1}{3} \cdot \frac{1}{2} \frac{1}{1}\right] \\
        &= \frac{3}{8} = 0.375
    \end{align*}

    We can see that even for small $n$ (in our case, $n=5$), the expected probabilities approach $\frac{1}{e} \approx 0.36788$.

\end{enumerate}
