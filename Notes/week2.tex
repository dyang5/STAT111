\newpage

\section{Continuous Random Variables}

\subsection{Intro to Continuous Random Variables}
Assumptions for a Poisson process in time include
\begin{itemize}
    \item Events are independent
    \item Rate of events through time is constant
    \item Events are not simultaneous
\end{itemize}

Suppose that $T_1, \dots, T_n$ are i.i.d.\ $\mathrm{Expo}(\lambda)$ random variables. \\

\begin{definition}
The \textbf{first order statistic}, defined to be the smallest of these, is 
$T_{(1)}$ follows the $\mathrm{Expo}(n \lambda)$ distribution.
\end{definition}

\begin{proof}
Consider the CDF of $T_{(1)}, P(T_{(1)} \leq t)$. Note that
\begin{align*}
    P(T_{(1)} \leq t) &= 1 - P(T_{(1)} > t) \\
    &= 1 - \left( e^{-\lambda t} \right)^n \\
    &= 1 - e^{-n\lambda t} 
\end{align*}

which is the CDF for an $\mathrm{Expo}(n \lambda)$ distribution.
\end{proof}


\subsection{The Differential Argument and the Gamma Distribution}

\begin{enumerate}[a)]
    \item By definition, 
    \[
        P(X \in [x, x+ dx)) = \int_{x}^{x+dx} f_x(y) \, dy.
    \]

    For small deviation $dx$, we can approximate the integral as $f_x(x)\left[ (x + dx) - x \right] = f_x(x)\, dx$. \\

    Note that this expression becomes exact when $dx$ approaches $0$: formally, we have
    \[
        \lim_{dx \to 0} \frac{P(X \in [x, x+dx))}{dx} = \lim_{dx \to 0} \frac{F(x+dx) - F(x)}{dx} = f_x(x)
    \]

    because the second expression is the limit form of the derivative of the CDF at $x$, which is precisely the value of the PDF at $x$: 
    $f_x(x)$. \\

    We can use the Differential argument to derive the Exponential density function. 
    Since the CDF of an Exponential variable at time $t$ measures the probability that an event has occured by time $t$, we can measure the probability
    that the first Poisson event occurs in the time interval $[t, t+dt)$ by subtracting the CDF at times $t+dt$ and time $t$:
    \begin{align*}
        P(\text{first occurence in } [t, t+dt)) &= \left( 1 - e^{-\lambda(t+dt)}\right) - \left( 1 - e^{-\lambda t} \right) \\
        &= e^{-\lambda t} - e^{-\lambda(t+dt)}
    \end{align*}

    Dividing this expression by $dt$ and taking the limit as $dt$ approaches $0$, we derive the Exponential variable density at time $t$:
    \[
        f_t(t) = \lim_{dt \to 0} \frac{e^{-\lambda t} - e^{-\lambda (t+dt)}}{dt}.
    \]

    Using L'Hopital's Rule, this becomes
    \begin{align*}
        f_t(t) &= \lim_{dt \to 0} \frac{e^{-\lambda t} - e^{-\lambda (t+dt)}}{dt} \\
        &= \lim_{dt \to 0} \frac{\lambda e^{-\lambda (t+dt)}}{1} \\
        &= \lambda e^{-\lambda t}.
    \end{align*}

    \item We can similarly use the differential argument to derive the Gamma$(k, \lambda)$ density; let $X$ be the time of the $k$th event for a Poisson process with rate $\lambda$.
    We can model the probability of $k-1$ events occurring before time $x$ using $\mathrm{Pois}(\lambda x)$, where $\lambda x$ represents the expected number of events to occur
    in $x$ units of time. Similarly, we can model the probability of $1$ event occurring in the interval $[x, x+dx)$ using $\mathrm{Pois}(\lambda dx)$. \\

    Thus, the probability of $k-1$ events occurring before time $x$ and a $k$th event occurring in $[x, x+dx)$ is 
    \begin{align*}
        &P(\text{$k - 1$ events before time $x$}) P(\text{$1$ event before time $t$}) \\
        &= \frac{(\lambda x)^{k-1} e^{-\lambda x}}{(k-1)!} \frac{(\lambda dx)^{1} e^{-\lambda dx}}{1!} \\
        &= \frac{\lambda^k e^{-\lambda (x+dx)}dx}{(k-1)!}x^{k-1}
    \end{align*}

    Dividing this probability by $dx$ and taking the limit as $dx$ approaches $0$, we find that
    \begin{align*}
        f_x(k, \lambda) &= \lim_{dx \to 0} \frac{\left( \frac{\lambda^k e^{-\lambda (x+dx)}dx}{(k-1)!}x^{k-1} \right) }{dx} \\
        &= \lim_{dx \to 0} \frac{\lambda^k e^{-\lambda (x+dx)}}{(k-1)!} x^{k-1} \\
        &= \frac{\lambda^k e^{-\lambda x}}{(k-1)!} x^{k-1}
    \end{align*}

    which matches the Gamma$(k, \lambda)$ density we expect. \\

    Note that if $X_1 \sim \mathrm{Gamma}(k_1, \lambda)$ and $X_2 \sim \mathrm{Gamma}(k_2, \lambda)$ are independent, intuitively, 
    $X_1 + X_2 \sim \mathrm{Gamma}(k_1 + k_2, \lambda)$, as we can think of $X_1$ and $X_2$ representing the sum for waiting times in two non-overlapping time intervals.
    We will reserve discussion for why the sum of two independent Gamma variables with different $\lambda$ values is not Gamma for part (d). 
    
    \item Recall the form of the Gamma PDF derived in part (b): 
    \[
        f_x(k, \lambda) = \frac{\lambda^k e^{-\lambda x}}{(k-1)!} x^{k-1}.
    \]

    Note that our derivation measures the average wait time for $k$ events to occur, where $k$ is an integer. 
    However, the Gamma Distribution also extends to values of $k$ which are not necessarily integers -- all positive real $k$, in fact. To understand the extension
    and subsequent definition of the Gamma Distribution, we introduce the notion of a Gamma Function, an extension of the factorial function for real numbers. 
    For integer $k$, $\Gamma(k) = (k-1)!$, which matches with the normalizing constant in our present PDF. \\

    In general, if $\alpha$ is a positive real value replacing our $k$ parameter, we define the PDF of $X \sim \mathrm{Gamma}(\alpha, \lambda)$ to be
    \[
        f_x(\alpha, \lambda) = \frac{\lambda^\alpha e^{-\lambda x}}{\Gamma(\alpha)} x^{\alpha-1}.
    \] 

    We can derive the definition of Gamma Function by forcing this PDF to integrate to $1$ as $x$ ranges from $0$ to $\infty$. Note that
    \[
        \int_0^{\infty} \frac{\lambda^\alpha e^{-\lambda x}}{\Gamma(\alpha)} x^{\alpha-1} \, dx = \frac{\lambda^\alpha}{\Gamma(\alpha)}\int_0^{\infty} e^{-\lambda x} x^{\alpha-1} \, dx.
    \]
    Apply the change of variables $y = \lambda x$. It follows that $x = \frac{y}{\lambda}$ so $dx = \frac{1}{\lambda} dy$. Our integral now becomes
    \[
        \frac{\lambda^\alpha}{\Gamma(\alpha)}\int_0^{\infty} e^{-\lambda x} x^{\alpha-1} \, dx = \frac{\lambda^\alpha}{\Gamma(\alpha)}\int_0^{\infty} e^{-y} \left( \frac{y}{\lambda} \right) ^{\alpha-1} \left(\frac{1}{\lambda }\, dy\right).
    \]

    Simplifying and setting the resulting expression to $1$ (since our PDF should integrate to $1$), we find that
    \[
        \frac{\int_0^{\infty} e^{-y} y^{\alpha - 1} \, dy}{\Gamma(\alpha)} = 1.
    \]

    This inspires our definition of the \textbf{Gamma Function}: 
    \[
        \boxed{\Gamma(\alpha) = \int_0^{\infty} e^{-y} y^{\alpha - 1} \, dy}
    \]

    which is defined for any positive real values of the parameter $\alpha$.

    \noindent\rule{\textwidth}{1pt} \\

    With the above definition of the Gamma Function, we can derive a nice recursive definition of the Gamma Function, 
    one which will illuminate its relationship with the factorial function. \\

    Note that by definition, $\Gamma(\alpha + 1) = \int_0^{\infty} e^{-y} y^{\alpha} \, dy$. \\

    Using Integration by Parts, we find that
    \begin{align*}
        \Gamma(\alpha + 1) &= \left[ y^\alpha (-e^{-y}) \right]_0^{\infty} - \int_0^{\infty} (-e^{-y}) \alpha y^\alpha \, dy \\
        &= 0 + \alpha \int_0^{\infty} (e^{-y}) y^\alpha \, dy \\
        &= \alpha \Gamma(\alpha).
    \end{align*}

    Since $\Gamma(1) = \int_0^{\infty} e^{-y} y^{1 - 1} \, dy = 1$ by definition, it follows recursively that $\Gamma(\alpha) = (\alpha - 1)!$.

    \item A final powerful property of the Gamma distribution is that for any constant $c > 0$, if $X \sim \mathrm{Gamma}(\alpha, \lambda)$, then
    $Y = cX \sim \mathrm{Gamma}\left(\alpha, \frac{\lambda}{c}\right)$. This property can be thought of as a changing in the units for the $\lambda$ parameter:
    intuitively, if $X$ measures the waiting time for $\alpha$ events to occur given a rate of $\lambda$ in some unit of time (e.g seconds), $Y=cX$ 
    may measure the waiting time for $\alpha$ events to occur given a rate of $\lambda$ in a new unit of time (e.g. minutes, where $c = \frac{1}{60}$). \\

    \textit{Note: This intuition, which relates the $\lambda$ parameter to a given unit of time, also gives us an idea of why
    the sum of two Gamma variables with different $\lambda$ values is not itself Gamma: we cannot sum the waiting times for events of
    processes measured in different units of time.} \\

    Suppose that $Y = cX$. Comparing the CDFs of $X$ and $Y$, we find that
    \[
        F_Y(y) = P(cX < y) = P \left( X < \frac{y}{c} \right) = F_X \left( \frac{y}{c} \right).
    \]
    Taking the derivative of both sides, we find a relationship between the PDFs. Since $y = cx$, we have
    \[
        f_Y(y) = \frac{1}{c} f_X \left( \frac{y}{c} \right) = \frac{1}{c} f_X(x).
    \]

    
    The scaling property itself can be proved by comparing the PDFs for $X$ and $Y$. Note that
    \[
        f_x(\alpha, \lambda) = \frac{\lambda^\alpha e^{-\lambda x}}{\Gamma(\alpha)} x^{\alpha-1}.
    \]

    Consider $\frac{1}{c} f_x(\alpha, \lambda) = \frac{1}{c}\frac{\lambda^\alpha e^{-\lambda x}}{\Gamma(\alpha)} x^{\alpha-1}$. We can rewrite this expression in a few equivalent forms:
    \begin{align*}
        \frac{1}{c} f_x(\alpha, \lambda) &= \frac{1}{c}\frac{\lambda^\alpha e^{-\lambda x}}{\Gamma(\alpha)} x^{\alpha-1} \\
        &= \left(\frac{1}{c}\right)^\alpha c^{\alpha - 1} \frac{\lambda^\alpha e^{-\lambda x}}{\Gamma(\alpha)} x^{\alpha-1} \\
        &= \left[\left(\frac{1}{c}\right)^\alpha \lambda^\alpha\right] \left(c^{\alpha - 1}x^{\alpha-1}\right) \frac{\lambda^\alpha e^{-\frac{\lambda}{x}(cx)}}{\Gamma(\alpha)} \\
        &= \left( \frac{\lambda}{c} \right)^\alpha (cx)^{\alpha-1} \frac{e^{-\frac{\lambda}{c}(cx)}}{\Gamma(\alpha)} 
    \end{align*}

    Note that the resulting expression is precisely the PDF for a $\mathrm{Gamma}\left( \alpha, \frac{\lambda}{c} \right)$ RV, so 
    \[
        Y = cX \sim \mathrm{Gamma}\left( \alpha, \frac{\lambda}{c} \right)        
    \] 
    as desired. 

    \subsection{The Uniform Distribution}

    \begin{definition}
    Let $F$ be the CDF for a continuous random variable. The \textbf{inverse function} $G(u)$ is defined
    to satisfy $G(u) = x$ if $F(x) = u$. \\

    Suppose that $U \sim \mathrm{Unif}(0, 1)$. Then  $X = G(U)$ has CDF $F$.
    \end{definition}

    \begin{remark}
    Let $U \sim \mathrm{Unif}(0, 1)$. Then  $X = G(U) \sim \mathrm{Expo}(\lambda)$. 
    \end{remark}

    \subsection{The Normal and Chi-Square Distributions}

    \begin{definition}[Chi-Square]
    The sum of $k$ independent squared standard normal variables follows a Chi-square distribution with $k$ degrees of freedom, denoted $\chi^2_(k)$. 
    \end{definition}

    \begin{remark}
    $\chi ^2_(1) \sim \mathrm{Gamma}\left( \frac{1}{2}, \frac{1}{2} \right)$. \\
    
    In general, $\chi ^2_(\nu) \sim \mathrm{Gamma}\left( \frac{\nu}{2}, \frac{1}{2} \right)$  due to the properties of a sum Gamma RVs. \\

    Furthermore, due to the scaling property of Gamma RVs, if $X \sim \mathrm{Gamma}(\alpha, \lambda)$, then $Y = 2\lambda X \sim \mathrm{Gamma}\left( \alpha, \frac{1}{2} \right)$
    which is equivalent to a $\chi ^2_{2\alpha}$ random variable.  
    \end{remark}

    \subsection{The Beta Distribution}

    \begin{definition}
    The \textbf{Beta Distribution} is the conjugate prior of the binomial distribution; if we assume a coin flip has a probability of success following a
    Beta distribution, the posterior distribution, after accounting for new data, is also a Beta distribution.
    \end{definition}

    \begin{remark}
    Let $V_1 \sim \mathrm{Gamma}(a, \lambda)$ and $V_2 \sim \mathrm{Gamma}(b, \lambda)$. Then
    \[
        X = \frac{V_1}{V_1 + V_2} \sim \mathrm{Beta}(a, b). 
    \]
    \end{remark}

    \begin{remark}
    Let $U_1, \dots, U_n$ be i.i.d.\ $\mathrm{Unif}(0, 1)$ random variables. Then the $k$th order statistic $U_k$ follows a $\mathrm{Beta}(k, n-k + 1)$ distribution for $k = 1, \dots , n$.   
    \end{remark}

    \subsection{The Beta, $F$, and $t$ distributions}
    \begin{definition}[Relationships between Gamma, Beta, and $F^*$ Distributions]
    Let $V_1 \sim \mathrm{Gamma}(a, \lambda)$, $V_2 \sim \mathrm{Gamma}(b, \lambda)$. \\
    
    Then $X = \frac{V_1}{V_1 + V_2} \sim \mathrm{Beta}(a, b)$. \\
    
    Then $R = \frac{V_1}{V_2} = \frac{X}{1-X}$ has PDF
    \[
        f_R(r) = \frac{\Gamma(a + b)}{\Gamma(a)\Gamma(b)} r^{a-1} \left( \frac{1}{1+r} \right)^{a+b}.  
    \]
    Furthermore, $Y = c \frac{V_1}{V_2} = cR \sim F^*(a, b, c)$ has PDF
    \begin{align*}
        f_Y(y) &= \frac{\Gamma(a + b)}{\Gamma(a)\Gamma(b)} \frac{y^{a-1} c^b}{(c+y)^{a+b}} I(y > 0) \\
        &\propto \frac{y^{a-1} c^b}{(c+y)^{a+b}} I(y > 0)
    \end{align*}
    for $a > 0$, $b > 0$, $c > 0$. This is known as the \textbf{$F^* (a, b, c)$}\textbf{ density.}
    \end{definition}

    \begin{definition}[F-distribution]
    The \textbf{$F$-distribution} is defined in terms of Chi-Square random variables: if $V_1 \sim \chi_(m_1)^2$ is independent of  $V_2 \sim \chi_(m_2)^2$, then
    \[
        F_{(m_1, m_2)} \sim \frac{\left( \frac{V_1}{m_1} \right) }{\left( \frac{V_2}{m_2} \right)}.
    \]
    \end{definition}

    \begin{remark}
    $F_{(n, m)} = F^*\left( \frac{n}{2}, \frac{m}{2}, \frac{m}{n} \right)$.
    \end{remark}

    \begin{definition}
    If $Z \sim N(0, 1)$ is independent of $V \sim \chi_{(m)}^2$, then 
    \[
        T = \frac{Z}{\sqrt{\frac{V}{m}} } \sim t_(m),
    \]
    the \textbf{t-distribution} with $m$ degrees of freedom. Furthermore, $T^2 \sim F_{(1, m)}$. 
    \end{definition}

\end{enumerate}
